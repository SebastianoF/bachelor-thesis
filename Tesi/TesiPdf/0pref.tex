
%PREFAZIONE

\chapter*{Introduzione}

Fin da quando frequentavo la scuola elementare sono sempre stato affascinato dal compasso e dalle figure geometriche che con esso si possono disegnare. Il divertimento nella costruzione dei poligoni regolari e una strana passione per il numero sette però mi portarono presto di fronte a una domanda: come mai per quanto ci provassi non riuscivo a costruire un poligono di sette lati?
Continuando gli studi scoprii che non ero il solo ad essermi posto questo tipo di problema e che anzi ne esistevano altri tre affini al mio, risalenti all'antichità. La loro storia merita di essere raccontata. \\


Era l'anno 429 a.C. quando Pericle, celebre stratega, morì durante un'epidemia di peste assieme ad un quarto del popolo ateniese.  
Durante tale pandemia, per placare l'ira degli dèi che si credeva ne fossero artefici, una delegazione di ateniesi andò ad interpellare l'oracolo di Apollo a Delo. Egli fu chiaro: se i sacerdoti avevano a cuore il futuro di Atene e dei suoi abitanti, allora l'altare cubico del dio della medicina avrebbe dovuto essere raddoppiato, senza che però ne fosse modificata la forma originale. 
I sacerdoti si misero immediatamente al lavoro e duplicarono la lunghezza degli spigoli del monumento. Inaspettatamente l'epidemia, anziché estinguersi, raggiunse i vertici della sua gravità. Pareva quindi che gli dèi non volessero più ragionare; fu allora interpellato l'uomo che sarà ricordato come l'unico saggio: Platone. Egli sostenne che Apollo avesse voluto punire i sacerdoti per la loro ignoranza, infatti il volume dell'altare cubico era stato moltiplicato per otto e non raddoppiato. A quel punto i migliori geometri, con i loro mezzi, cioè riga e compasso, si misero alla ricerca di una soluzione. Secondo la leggenda fu questa l'origine del problema della  \begin{bfseries}duplicazione del cubo\end{bfseries}, noto anche come \lq\lq problema di Delo\rq\rq. Nonostante gli sforzi, però, nessuno riuscì mai a risolverlo, cioè a trovare la lunghezza del lato che avrebbe reso il volume del cubo doppio di quello iniziale \cite{Boyer}.\\

Durante la stessa epoca, sempre ad Atene, era stato posto un altro problema: \begin{bfseries}la trisezione dell'angolo\end{bfseries}, nel quale si richiede il taglio di un angolo in tre angoli interni di uguale ampiezza \cite{sito3}. Archimede fu il primo a trovare una soluzione, che prevedeva però l'uso di una riga graduata. Coloro i quali perseverarono nella ricerca di una soluzione con il solo uso della riga non graduata e del compasso non giunsero mai ad un risultato.\\

Ecco il terzo celebre problema affine ai due già proposti: \begin{bfseries}la quadratura del cerchio\end{bfseries}. Esso consiste nella ricerca di un procedimento per costruire un quadrato con la stessa area di un cerchio dato, con il solo aiuto di riga e compasso. Analogamente si può considerare il problema di trovare un segmento di lunghezza pari a quella di una circonferenza data. La ricerca della soluzione esatta è stata un inutile sforzo per i matematici dei secoli successivi, fino ad essere considerata la metafora di un'impresa disperata, al punto che Dante, al cospetto della visione divina rappresentata nelle sue terzine, si paragona al geometra che tenta di quadrare il cerchio, aggrappato alla fede nell'esistenza di una soluzione \cite{Dante}:

\begin{verse}
\vspace*{0.5cm}
\emph{ 
Qual è 'l geometra che tutto s'affige \\
per misurar lo cerchio, e non ritrova,\\
pensando, quel principio ond'elli indige, \\
\vspace*{0.3cm}
tal era io a quella vista nova:  \\
veder voleva come si convenne\\
l'imago al cerchio e come vi s'indova; \\
\vspace*{0.3cm}
ma non eran da ciò le proprie penne: \\
se non che la mia mente fu percossa \\
da un fulgore in che sua voglia venne. }
\vspace*{0.5cm}
\end{verse}

Rimasi sorpreso nello scoprire che, per più di $2000$ anni di storia, gli sforzi per risolvere i tre problemi esposti, nonché la costruzione di alcuni poligoni, come quello di $7$ lati che cercai con ostinazione, furono tutti vani. Non solo infatti questi non sono risolubili con il solo uso di riga e compasso, ma la dimostrazione rigorosa dell'impossibilità arrivò solo intorno alla fine del diciannovesimo secolo; tale dimostrazione è l'argomento della mia tesi.








