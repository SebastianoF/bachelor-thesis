%questo va messo nell'appendice!

\chapter*{Appendice}



I lettori di questa tesi si saranno chiesti come mai così tanti matematici hanno trascorso molto tempo a cercare una soluzione dei tre problemi classici, usando esclusivamente riga e compasso, nonostante avessero a disposizione una soluzione trovata con mezzi più moderni. In queste poche righe, vorrei provare a spiegare che la loro non è stata solo la ricerca della soddisfazione personale, ma è conseguenza di motivazioni filosofiche e inaspettatamente pratiche. \\

Come è noto, negli Elementi di Euclide, la riga e il compasso sono alla base di un sistema assiomatico, che una volta finito nelle mani dei matematici successivi è stato ammirato per la sua forma, eleganza e per la sua completezza. \\
In verità le lacune degli Elementi sono molteplici; per esempio nella costruzione del triangolo equilatero si usa una proprietà che non viene mai ne dimostrata ne espressa chiaramente, cioè che due circonferenze, con centro sui diversi estremi di un segmento e aventi per raggio il segmento stesso, abbiano intersezione non vuota \cite{Shea}. \\
Basandosi sull'osservazione di questa ed altre lacune, lo studioso Lucio Russo \cite{Russo}, sostiene che gli Elementi non sono altro che una sorta di manuale di istruzioni per utilizzare il più potente calcolatore dell' epoca: la riga e il compasso.
L'idea che ogni figura geometrica pensabile sia costruibile e quindi misurabile direttamente con questi due oggetti ci porterebbe a credere che i greci avessero sviluppato un sistema completo, elegante e soprattutto definitivo.\\

A questo punto l'impossibilità di risolvere i tre problemi classici, arrivata per via algebrica, ha dimostrato la necessità di affrontare nuovi orizzonti. Ha dato quindi un contributo fondamentale a rendere, già nella seconda metà dell'ottocento, non del tutto soddisfacente la visione della geometria fornita dall'opera di Euclide. 

