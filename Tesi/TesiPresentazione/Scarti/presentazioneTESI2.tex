\documentclass[11pt]{beamer}
\usepackage[latin1]{inputenc}
\usepackage[italian]{babel}
\usepackage{amssymb,amsmath,amsfonts,amsthm,mathrsfs}
\usepackage{graphicx}
\usepackage{verbatim}
\usepackage{makeidx}
\pagestyle{empty}

\newtheorem{definizione}{Definizione}[section]
\newtheorem{osservazione}{Osservazione}[section]
\newtheorem{teorema}{Teorema}[section]
\newtheorem{prop}{Propriet\`{a}}[section]
\newtheorem{proposizione}{Proposizione}[section]
%\newtheorem{lemma}{Lemma}[section]
\newtheorem{convenzione}{Convenzione}[section]
\newtheorem{der1}{Derivazione del limite}[section]
\newtheorem{der2}{Formula di Leibnitz}[section]
\newtheorem{der3}{Derivata della funzione composta}[section]

\usetheme{Copenhagen}
%\usecolortheme{albatros}


\title{TESI DI LAUREA}
\author{Federica Narciso}
\date{}
\subtitle{PARTI FINITE}
\institute{Università degli studi di Torino \\Facoltà di Scienze MM.FF.NN.\\ Corso di laurea in Matematica}




\begin{document}


\begin{frame}
 \maketitle
 \begin{center}
 \includegraphics<1>[scale=0.5]{unito}
 \end{center}
\end{frame}







%%%%%%%%%%%%%%%%%%%%%%%%%%%INTRODUZIONE
\begin{frame}
\frametitle{Introduzione}
Sia $g$ una funzione reale di variabile reale. Nostro scopo è dare un significato a $\int_a ^b g(x)\,dx,$ dove $(a,b)$ è un intervallo finito, nel caso in cui la definizione di integrale improprio di Riemann non sia sufficiente. Supporremo, in particolare, che nel punto $a$ tale integrale sia \textit{divergente.} Secondo un'idea introdotta da Hadamard nel 1932 si osserva che, sotto certe ipotesi,
$$\int_{a+\epsilon}^b g(x)\, dx $$
E' somma di alcune quantità che divergono per $\epsilon\to 0$ e di altre che, al contrario, per $\epsilon\to 0$ convergono. Indicate queste ultime con $F(\epsilon),$ sarà
$$\lim_{\epsilon\to 0}F(\epsilon)$$
la \textit{parte finita} dell'integrale.
\end{frame}




%%\begin{frame}
%%\frametitle{Introduzione}
%%%Inoltre:
%%%%%%%\begin{itemize}
%%\item Si studieranno le propriet‡ fondamentali di tali parti finite.
%%\item  Si vedranno i concetti principali inerenti la teoria delle distribuzioni.
%%%\item Si vedr‡ quando Ë possibile definire
 %%%$$Pf.\int_{-\infty}^{+\infty} g(x)v(x)\,dx.$$ Tale parte
%%%finita, chiamata pseudofunzione, Ë una distribuzione.
%%\item Si daranno le regole per il calcolo delle parti finite e delle loro derivate.
%%\end{itemize}
%%%%\end{frame}












%%%%%%%%%%%%%%%%%%%%%%%%%%%%%%%%%%%%%%% PARTI FINITE
\begin{frame}
\frametitle{Le Parti Finite}
%\begin{block}
\begin{definizione}

Sia $g:(a,b)\subset\mathbf{R}\rightarrow\mathbf{R}$ con $(a,b)$ intervallo finito.
Sia la funzione $g$ integrabile su ogni intervallo finito $(a+\epsilon,b)$ per ogni $\epsilon>0$, ma non integrabile su tutto $(a,b)$.\\
Supponiamo che $g$ sia rappresentabile come somma di una funzione $h$ integrabile su $(a,b)$ e di un polinomio in $\frac{1}{x-a}$, ovvero:
\begin{equation}
g(x)=P\left( \frac{1}{x-a} \right) + h(x) = \sum_\nu \frac{A_\nu}{(x-a)^{\lambda_\nu}} + h(x)
\end{equation}
con $A_\nu \in \mathbf{R}, \lambda_\nu \in \mathbf{R}.$\\
\noindent Allora si definisce la \textbf{parte finita dell'integrale} $\mathbf{\int_a^b g(x)\,dx}$ come:

\begin{displaymath}
F = Pf.\int_a^b g(x) \,dx= -\sum_\nu \frac{A_\nu}{\lambda_\nu -1}\left( \frac{1}{b-a}\right) ^{\lambda_\nu -1} +\int_a^b h(x) \,dx
\end{displaymath}

\end{definizione}
\end{frame}

\begin{frame}
\frametitle{Le Parti Finite}
\begin{definizione}
per $\lambda_\nu$ non interi,
\begin{align}
F &= Pf.\int_a^b g(x) \,dx = \notag \\
& =-\sum_{\nu\ne 1} \frac{A_\nu}{\lambda_\nu -1}\left( \frac{1}{b-a}\right) ^{\lambda_\nu -1}+A_1\log (b-a) +\int_a^b h(x) \,dx
\notag
\end{align}
per $\lambda_\nu$ interi, dove $A_1$ è il coefficiente del termine di primo grado del polinomio P.


\end{definizione}

%\end{block}
\end{frame}


%%%%%%%%%%%%%%%%%%%%%%%%%%%%%%%%%%% BIBLIOGRAFIA

\begin{frame}
\frametitle{Bibliografia}
\begin{thebibliography}{4}


\bibitem{I.M. Gel'fand G.E.Shilov}
I.M. Gel'fand \& G.E. Shilov, \emph{Generalized functions}, Academic Press, 1964.


\bibitem{bouix}
M. Bouix, \emph{Les fonctions g\'{e}n\'{e}ralis\'{e}es ou distributions}, Masson Editore, 1964.


\bibitem{tricomi}
F.G. Tricomi, \emph{Istituzioni di analisi superiore}, Cedam, 1970.



\bibitem{bureau}
F.J. Bureau \emph{Divergent integrals and partial differential equations}, Communications on Pure and Applied Mathematics, 1955.
\end{thebibliography}
\end{frame}





\end{document} 